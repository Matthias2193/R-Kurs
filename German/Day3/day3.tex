\documentclass[aspectratio = 169]{chariteBeamer}
\usepackage[german]{babel} %% english
\usepackage[utf8]{inputenc}
\usepackage[T1]{fontenc}
\usepackage{hyperref}
\usepackage{blkarray}
\tikzset{>=latex}
\usepackage[edges]{forest}
\usetikzlibrary{positioning}
\usepackage{biostat}
\setbeamertemplate{caption}[numbered]
\let\qed\relax
\forestset{declare toks={elo}{}}
\graphicspath{{figures/}}
%% ================================================================== %% 

\author[L. Mödl, M. Becher, E. Sprünken]{Lukas Mödl, Matthias Becher, Erin Sprünken} 
\title{Tag 3 -- Statistische Tests \& Regression} 
\date[]{Aktualisiert: \today}
\place{R-Kurs}
\email{biometrie-rkurs@charite.de}

%% ================================================================== %% 

\hyphenation{Sam-ples}
\begin{document}

%% ================================================================== %%
%% ================================================================== %%

\begin{frame}[plain]
    \titlepage%
\end{frame}
\frame{\tableofcontents}



\section{Statistische Tests}

\begin{frame}[fragile]{Statistische Tests in R}
	\begin{itemize}
		\item t-Test = \verb+t.test()+
		\item Chi-Quadrat Test = \verb+chisq.test()+
		\item Wilcoxon-Mann-Whitney-Test = \verb+wilcox.test()+
		\item Fisher Test = \verb+fisher.test()+
		\item McNemar's Test = \verb+mcnemar.test()+
		\item Binomial Test = \verb+binom.test()+
		\item ...
	\end{itemize}
\end{frame}


\begin{frame}[fragile]{t-Test}
\verb+t.test(x,...)+\\
Parameter:
	\begin{itemize}
		\item x = Ein Vektor mit Daten
		\item y = Ein optionaler Vektor mit Daten, falls man zwei Gruppen vergleichen möchte
		\item alternative = c("two.sided", "less", "greater")
		\item mu = Der angenommene Mittelwert unter der Nullhypothese
		\item paired = c(TRUE, FALSE)
	\end{itemize}
\end{frame}

\begin{frame}[fragile]{Beispiel t-Test:}	
	\begin{center}
		\includegraphics{OneSampleTtest}
	\end{center}
\end{frame}

\begin{frame}[fragile]{Beispiel t-Test:}	
	\begin{center}
		\includegraphics{TwoSampleTtest}
	\end{center}
Anmerkung: Per default nimmt R beim Zwei-Stichproben-t-Test ungleiche Varianz an
\end{frame}

\begin{frame}[fragile]{Chi-Quadrat Test:}	
	\verb+chisq.test()+ \\

	Beispiel: \\
	\begin{columns}[T]
		\begin{column}{0.5\textwidth}
			\includegraphics[width=7.5cm]{tabledata}
		\end{column}
		\begin{column}{0.5\textwidth}
			\includegraphics[width=7.5cm]{chisq}
		\end{column}
	\end{columns}
\end{frame}


%% ================================================================== %%
\section{Regressionsanalysen}

\begin{frame}[fragile]{Formeln in R}
	Um eine Regression durchzuführen müssen wir der Funktion sagen, welche Spalten in unseren Daten die unabhängigen Variablen sind und welche Spalte die abhängige Variable ist. Dafür gibt	es in R die Formelschreibweise:
	\begin{itemize}
		\item Nur bestimmte Variablen sollen in der Regression verwendet werden:
			\begin{center}
				Y\textasciitilde X1 + X2 + X3 + ... 
			\end{center}
		\item Alle Variablen im Datensatz sollen in der Regression verwendet werden:
			\begin{center}
				Y\textasciitilde.
			\end{center}
	\end{itemize}
\end{frame}



\begin{frame}[fragile]{Lineare Regression}
	\begin{itemize}
		\item \verb"model <- lm(Weight~Age+Sex+Height+Klinik, data =data)"
		\item \verb+summary(model)+
		\item Anmerkung: "0 +" \text{am} Anfang der Formel führt zu einer Regression ohne Intercept 
	\end{itemize}
			
	\begin{center}
		\includegraphics[height=3.75cm]{LinearRegressionSummary}
	\end{center}
\end{frame}

\begin{frame}[fragile]{Lineare Regression Plot}
	\begin{columns}[T]
		\begin{column}{0.6\textwidth}
			\begin{itemize}
				\item \verb+plot(data$Height,data$Weight)+
				\item \verb+abline(model)+
			\end{itemize}
		\end{column}
		\begin{column}{0.4\textwidth}
			\includegraphics[height=6.5cm]{LinearRegressionPlot}
		\end{column}
	\end{columns}
\end{frame}

\begin{frame}[fragile]{Kaplan-Meier Plot}
	\begin{itemize}
		\item \verb+library(survival)+ \\ \verb+data_vet <- veteran+ \\ \verb+km_fit <- survfit(Surv(time, status) ~ 1, data=data_vet)+ \\ \verb+plot(km_fit)+
	\end{itemize}
			
	\begin{center}
		\includegraphics[height=5cm]{KM1}
	\end{center}
\end{frame}

\begin{frame}[fragile]{Kaplan-Meier Plot}
	\begin{itemize}
		\item \verb+library(survminer)+ \\ \verb+ggsurvplot(km_fit)+
	\end{itemize}
			
	\begin{center}
		\includegraphics[height=5.5cm]{KM2}
	\end{center}
\end{frame}

\begin{frame}[fragile]{Kaplan-Meier Plot}
	\begin{itemize}
		\item \verb+km_fit <- survfit(Surv(time, status) ~ trt, data=data_vet)+ \\ \verb+ggsurvplot(km_fit)+
	\end{itemize}
			
	\begin{center}
		\includegraphics[height=5.5cm]{KM3}
	\end{center}
\end{frame}

\begin{frame}[fragile]{Logistische Regression}
	\begin{itemize}
		\item \verb+model <- glm(y~., data = logistic_data, family = binomial)+
		\item \verb+summary(model)+
	\end{itemize}
			
	\begin{center}
		\includegraphics[height=4cm]{LogisticRegressionSummary}
	\end{center}
\end{frame}

\begin{frame}[fragile]{One-Way ANOVA}
	\begin{itemize}
		\item \verb+model <- aov(formula, data)+
	\end{itemize}	
	\begin{center}
		\includegraphics{OneWay}
	\end{center}
\end{frame}

\begin{frame}[fragile]{Two-Way ANOVA}
	\begin{center}
		\includegraphics{TwoWay}
	\end{center}
\end{frame}

\begin{frame}[fragile]{Interaction ANOVA}
	\begin{center}
		\includegraphics{Interaction}
	\end{center}
\end{frame}




%% ================================================================== %%

	


\section{\textsf R Pakete}


\begin{frame}[fragile]{Installation weiterer \textsf R Pakete}
    Jede \textsf R Umgebung installiert und lädt standardmäßig die Pakete \texttt{base}, \texttt{stats}, \texttt{datasets}, \texttt{methods} und \texttt{graphics}.\bigskip
    \begin{itemize}
        \item Installation weiterer Pakete mit:
        \begin{verbatim}
        install.packages("name-des-pakets", dependencies = TRUE)
        \end{verbatim}
        \item Bei jedem Start von \textsf R muss das Paket, wenn es verwendet werden soll, geladen werden:
        \begin{verbatim}
        library("name-des-pakets")
        \end{verbatim}
        \item Aktualisieren der Pakete mit:
        \begin{verbatim}
        update.packages()
        \end{verbatim}
    \end{itemize}
\end{frame}

\begin{frame}[fragile]{Beispiel: Installation und Laden des \textsf R Pakets \texttt{MASS}}
    \begin{center}
        \includegraphics[width=\textwidth, keepaspectratio]{pakete.png}
    \end{center}
\end{frame}


\begin{frame}[fragile]{Empfehlenswerte Pakete}
	\begin{itemize}
		\item \verb+MatchIt+ für Propensity Score Matching 
	          \item \verb+MASS+ für Negativ-binomiale Regression \\
	          \item \verb+lmer+ bzw. \verb+lme4+ für Mixed-Models \\
	          \item \verb+pwr+ für Power-Analyse und insbesondere zur Fallzahlplanung \\
	          \item \verb+ggplot2+ für schöne Plots \\
	          \item \verb+haven+ für das Einlesen von \verb+.sav+-Dateien (SPSS) \\ 
	          \item ...
	\end{itemize}
\end{frame}
	



\end{document}
%% ================================================================== %%
%% ================================================================== %% 
