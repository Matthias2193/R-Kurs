\documentclass[xcolor=dvipsnames, aspectratio = 169]{beamer}
\usepackage[english]{babel} %% english
\usepackage[utf8]{inputenc}
\usepackage[T1]{fontenc}
\usepackage{include/chariteBeamer}
\usepackage{hyperref}
\input{include/header2}
\tikzset{>=latex}
%\usepackage{amsmath}
%\usetikzlibrary{shadows}
\usepackage[edges]{forest}
\usetikzlibrary{positioning}
\usepackage{biostat}
\setbeamertemplate{caption}[numbered]
\let\qed\relax
\forestset{declare toks={elo}{}}
%% ================================================================== %% 

\author[L. Mödl, M. Becher, E. Sprünken]{Lukas Mödl, Matthias Becher, Erin Sprünken} 
\title{R-Kurs: Tag 3} 
\date[]{\today}

%% ================================================================== %% 
\setbeamercolor*{mycol}{bg=chariteGray, fg=chariteBlue}

\hyphenation{Sam-ples}
\begin{document}

%% ================================================================== %%
%% ================================================================== %%
\setbeamertemplate{footline}{\begin{tikzpicture}
    \node [inner sep=0pt, anchor=east] (0,0) {
      \includegraphics[width=\paperwidth,height=0.7cm]{include/charite_footer}};
    \node [inner sep=0pt, anchor=east] at (-0.5ex,-0ex){};
\end{tikzpicture}}

\setbeamertemplate{headline}{
%\leavevmode
\hspace{-0.49em}\hbox{
	\begin{beamercolorbox}[wd=1.02\paperwidth,ht=2.25ex,dp=1ex,left]{mycol}%
    \usebeamerfont{section in head/foot}
  \end{beamercolorbox}%
}}
{
  \usebackgroundtemplate{ \hspace{-0.5em}\begin{tikzpicture}
  \node[opacity=0.7, anchor=south] (0,0) {\includegraphics[height=\paperheight, width=1.04\paperwidth]{include/frontmatter.pdf}};
  \end{tikzpicture}
} 
%\frame{\titlepage}
\begin{frame}
\centering
	\vspace{4em}
	{\Large \textcolor{chariteBlue}{\inserttitle}}\\
	 \vspace{1em}
	{\Large \textcolor{black}{\insertauthor \\}} 
	\vspace{2em}
	{\footnotesize \textcolor{black}{\insertinstitute \\\vspace{1em} \insertdate}} 
	\vspace{0em}
	\begin{figure}[h!]
		\includegraphics[width=5cm]{include/Charite_Logo.png}
	\end{figure}
%	\pgfuseimage{frontUnilogo}
\end{frame}
}
%% ================================================================== %%

\setbeamertemplate{footline}{\begin{tikzpicture}
    \node [inner sep=0pt, anchor=east] (0,0) {
      \includegraphics[width=\paperwidth,height=0.7cm]{include/charite_footer}};
    \node [inner sep=0pt, anchor=east] at (-0.5ex,-0ex) {\tiny \insertframenumber{}$\,$|$\,$\inserttotalframenumber};
\end{tikzpicture}}

\setbeamertemplate{headline}{%
%\leavevmode%
\hspace{-0.49em}\hbox{
	\begin{beamercolorbox}[wd=.68\paperwidth,ht=2.25ex,dp=1ex,left]{mycol}%
    \usebeamerfont{section in head/foot}\hspace*{1em}
  \end{beamercolorbox}%
  \begin{beamercolorbox}[wd=.20\paperwidth,ht=2.25ex,dp=1ex,right]{mycol}%
    \usebeamerfont{author in head/foot}\insertshortauthor
  \end{beamercolorbox}%
  \begin{beamercolorbox}[wd=.14\paperwidth,ht=2.25ex,dp=1ex,center]{mycol}%
    \usebeamerfont{date in head/foot}\insertdate
  \end{beamercolorbox}%
  }
}


\frame{\tableofcontents}

\setbeamertemplate{headline}{%
%\leavevmode%
\hspace{-0.49em}\hbox{
	\begin{beamercolorbox}[wd=.68\paperwidth,ht=2.25ex,dp=1ex,left]{mycol}%
    \usebeamerfont{section in head/foot}\hspace*{1em}\thesection. \  \insertsectionhead
  \end{beamercolorbox}%
  \begin{beamercolorbox}[wd=.20\paperwidth,ht=2.25ex,dp=1ex,right]{mycol}%
    \usebeamerfont{author in head/foot}\insertshortauthor
  \end{beamercolorbox}%
  \begin{beamercolorbox}[wd=.14\paperwidth,ht=2.25ex,dp=1ex,center]{mycol}%
    \usebeamerfont{date in head/foot}\insertdate
  \end{beamercolorbox}%
  }
}

\section{Statistische Tests}
\begin{frame}[fragile]{T-Test}
\verb+t.test(x,...)+\\
Parameter:
	\begin{itemize}
		\item x = Ein Vektor mit Daten
		\item y = Ein optionalte Vektor mit Daten, falls man zwei Gruppen vergleichen möchte
		\item alternative = c("two.sided", "less", "greater")
		\item mu = Der wahre Mittelwert
		\item paired = c(TRUE, FALSE)
	\end{itemize}
\end{frame}

\begin{frame}[fragile]{Beispiel T-Test:}	
	\begin{center}
		\includegraphics{OneSampleTtest}
	\end{center}
\end{frame}

\begin{frame}[fragile]{Beispiel T-Test:}	
	\begin{center}
		\includegraphics{TwoSampleTtest}
	\end{center}
\end{frame}

\begin{frame}[fragile]{Chi-Quadrat Test:}	
	\verb +chisq.test()+ \\

	Beispiel: \\
	\begin{columns}[T]
		\begin{column}{0.4\textwidth}
			\includegraphics{tabledata}
		\end{column}
		\begin{column}{0.6\textwidth}
			\includegraphics{chisq}
		\end{column}
	\end{columns}
\end{frame}


\begin{frame}[fragile]{Weitere statistische Tests}
	\begin{itemize}
		\item Wilcoxon-Mann-Whitney-Test = \verb+wilcox.test()+
		\item Fisher Test = \verb+fisher.test+
		\item McNemar's Test = \verb+mcnemar.test()+
		\item Binomial Test = \verb+binom.test()+
		\item ...
	\end{itemize}
\end{frame}

%% ================================================================== %%
\section{Regressionsanalysen}

\begin{frame}[fragile]{Formeln in R}
	Um eine Regression durchzuführen müssen wir der Funktion sagen, welche Spalten in unseren Daten die abhängigen Variablen sind und welche Spalte der abhängige Variable ist. Dafür gibt 	es in R die Formelschreibweise:
	\begin{itemize}
		\item Nur bestimmte Variablen sollen in der Regression verwendet werden:
			\begin{center}
				Y\textasciitilde X1 + X2 + X3 + ... 
			\end{center}
		\item Alle Variablen sollen in der Regression verwendet werden:
			\begin{center}
				Y\textasciitilde.
			\end{center}
	\end{itemize}
\end{frame}



\begin{frame}[fragile]{Lineare Regression}
	\begin{itemize}
		\item \verb+model <- glm(y~., data = linear_data)+
		\item \verb+summary(model)+
	\end{itemize}
			
	\begin{center}
		\includegraphics[height=4cm]{LinearRegressionSummary}
	\end{center}
\end{frame}

\begin{frame}[fragile]{Lineare Regression Plot}
	\begin{columns}[T]
		\begin{column}{0.6\textwidth}
			\begin{itemize}
				\item \verb+plot(linear_data$x, linear_data$y)+
				\item \verb+abline(model)+
			\end{itemize}
		\end{column}
		\begin{column}{0.4\textwidth}
			\includegraphics[height=6.5cm]{LinearRegressionPlot}
		\end{column}
	\end{columns}
\end{frame}

\begin{frame}[fragile]{Logistische Regression}
	\begin{itemize}
		\item \verb+model <- glm(y~., data = logistic_data, family = binomial)+
		\item \verb+summary(model)+
	\end{itemize}
			
	\begin{center}
		\includegraphics[height=4cm]{LogisticRegressionSummary}
	\end{center}
\end{frame}

\begin{frame}[fragile]{One-Way ANOVA}
	\begin{itemize}
		\item \verb+model <- aov(formula, data)+
	\end{itemize}	
	\begin{center}
		\includegraphics{OneWay}
	\end{center}
\end{frame}

\begin{frame}[fragile]{Two-Way ANOVA}
	\begin{center}
		\includegraphics{TwoWay}
	\end{center}
\end{frame}

\begin{frame}[fragile]{Interaction ANOVA}
	\begin{center}
		\includegraphics{Interaction}
	\end{center}
\end{frame}




%% ================================================================== %%
\section{Apply-Familie}
\begin{frame}[fragile]{Apply-Familie}
	Die Apply-Familie ist eine Reihe von Funktion in R, die es uns erlaubt eine Funktion auf auf mehrere verschiedenen Inputs nacheinander anzuwenden. Zum Beispiel auf alle Zeilen oder Spalten einer Matrix oder alle Elemente einer Liste. Die verschiedenen Apply-Funktionen sind:  
	\begin{itemize}
		\item \verb+apply()+
		\item \verb+lapply()+
		\item \verb+sapply()+
		\item \verb+tapply()+
	\end{itemize}
\end{frame}

\begin{frame}[fragile]{apply()}
	Mit apply() können wir Funktionen auf alle Zeilen oder Spalten eines DataFrames oder einer Matrix anwenden um zum Beispiel alle Spaltensummen zu berechnen. Die Grundform der Funktion ist:
	\begin{itemize}
		\item \verb+apply(data, margin, function)+
	\end{itemize}
	Zum Beispiel:
	\begin{columns}[T]
		\begin{column}{0.5\textwidth}
			\includegraphics{Apply}
		\end{column}
		\begin{column}{0.4\textwidth}
			\[
			\begin{bmatrix}
			    1 &2 &3 &4\\
			    5 &6 &7 &8\\
			    9 &10 &11 &12\\
			    13 &14 &15 &16\\
  			\end{bmatrix}
			\]
		\end{column}
	\end{columns}
	
\end{frame}

\begin{frame}[fragile]{lapply()}
	lapply() führt eine Funktion auf jedes Element eines DataFrames, einer Matrix, eines Vektors oder einer Liste aus. Das "l" in lapply() steht dabei für "list" und bezieht sich darauf, dass lapply() immer eine Liste zurück gibt.
	\begin{itemize}
		\item \verb+lapply(object, function)+
	\end{itemize}
	Zum Beispiel:\\
	\begin{center}
		\includegraphics{lapply}
	\end{center}
	
	
\end{frame}

\begin{frame}[fragile]{sapply()}
	sapply() macht im Grunde das gleiche wie lapply(). Der Unterschied ist, dass sapply() einen Vektor oder eine Matrix zurück gibt, anstatt einer Liste:
	\begin{itemize}
		\item \verb+sapply(object, function)+
	\end{itemize}
	Zum Beispiel:\\
	\begin{center}
		\includegraphics{sapply}
	\end{center}
\end{frame}

\begin{frame}[fragile]{tapply()}
	tapply() erlaubt es uns auf Grundlage von factor levels Gruppenzusammenfassungen zu erstellen. :
	\begin{itemize}
		\item \verb+tapply(object, index, function)+
	\end{itemize}
	Zum Beispiel:\\
	\begin{center}
		\includegraphics{tapply}
	\end{center}
	
\end{frame}
\end{document}
%% ================================================================== %%
%% ================================================================== %% 
